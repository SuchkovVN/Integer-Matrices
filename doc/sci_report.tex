\documentclass[12pt]{article}      
\usepackage[T2A]{fontenc}
\usepackage[utf8x]{inputenc}
\usepackage[russian]{babel}
\usepackage{amsmath}
\usepackage{amsthm}
\usepackage{amssymb}
\usepackage{listings}
\usepackage[pdftex]{graphicx}
\usepackage{caption}
\usepackage[a4paper,left=3cm,right=3cm,
top=2cm,bottom=2cm,bindingoffset=0cm]{geometry}

\captionsetup[figure]{name=Рисунок}

\newtheorem{definition}{Определение}[section]
\newtheorem{example}{Пример}[section]
\newtheorem{statement}{Утверждение}[section]
\newtheorem{theorem}{Теорема}
\newtheorem{lemm}{Лемма}[section]

\usepackage{cite}
\makeatletter
\renewcommand{\@biblabel}[1]{#1.}
\makeatother

\usepackage{indentfirst}
\usepackage{hyperref}

\geometry{left = 2.5cm, right = 1.5cm, top = 2cm, bottom = 2cm}

\begin{document}

\begin{center}
Министерство науки и высшего образования Российской Федерации \\
Федеральное государственное автономное образовательное учреждение \\
высшего образования\\ 
\textbf{«Национальный исследовательский \\ Нижегородский государственный университет \\
им. Н.И. Лобачевского (ННГУ)»}\\[1.5cm]
\textbf{Институт информационных технологий, математики и механики\\
Кафедра алгебры, геометрии и дискретной математики}\\[5.5cm]

\textbf{\large ОТЧЕТ} \\
по \textbf{научно-исследовательской} работе \\[0.6cm]

на тему:\\
\textbf{\large Структура множества целочисленных решений матричного уравнения $AX = XB$}\\[3.7cm]

\begin{flushright}
\begin{minipage}{0.52\textwidth}
\begin{flushleft}
\textbf{Выполнил:} \\
студент группы 3825М1ПМкн1 \\
Сучков В.Н.\\[0.3cm]
Подпись:\\[0.3cm]
\textbf{Научный руководитель:}\\
доцент каф. АГДМ, к.ф.-м.н.,\\  Сидоров С.В.\\[0.3cm]
Подпись:\\[0.3cm]
\end{flushleft}
\end{minipage}
\end{flushright}

\vfill

Нижний Новгород \\
2025

\thispagestyle{empty}
\end{center}

\newpage
\tableofcontents
\thispagestyle{empty}
\newpage
\setcounter{page}{1}
\section{Введение}


\end{document}