\documentclass[11pt, a4paper]{article}      
\usepackage[T2A]{fontenc}
\usepackage[utf8x]{inputenc}
\usepackage[russian]{babel}
\usepackage{csquotes}
\usepackage{amsmath}
\usepackage{amsthm}
\usepackage{enumitem}
\usepackage{amssymb}
\usepackage{listings}
\usepackage[pdftex]{graphicx}
\usepackage{caption}
\usepackage[a4paper,left=2cm,right=2cm,
top=2cm,bottom=2cm,bindingoffset=0cm]{geometry}
\usepackage[colorlinks=true, linkcolor=black, urlcolor=blue, citecolor=black]{hyperref}
\usepackage{indentfirst}
\usepackage[nottoc]{tocbibind}
\usepackage{arydshln}

\captionsetup[figure]{name=Рисунок}

\newtheorem{definition}{Определение}
\newtheorem{example}{Пример}
\newtheorem{statement}{Утверждение}
\newtheorem{theorem}{Теорема}
\newtheorem{lemm}{Лемма}

\def\intmat#1#2{\mathbb{Z}^{#1 \times #2}}
\def\ratmat#1#2{\mathbb{Q}^{#1 \times #2}}

\begin{document}

\begin{center}
Министерство науки и высшего образования Российской Федерации \\
Федеральное государственное автономное образовательное учреждение \\
высшего образования\\ 
\textbf{«Национальный исследовательский \\ Нижегородский государственный университет \\
им. Н.И. Лобачевского (ННГУ)»}\\[1.5cm]
\textbf{Институт информационных технологий, математики и механики\\
Кафедра алгебры, геометрии и дискретной математики}\\[5.5cm]

\textbf{\large ОТЧЕТ} \\
по \textbf{научно-исследовательской} работе \\[0.6cm]

на тему:\\
\textbf{\large Структура множества целочисленных решений матричного уравнения $AX = XB$}\\[3.7cm]

\begin{flushright}
\begin{minipage}{0.52\textwidth}
\begin{flushleft}
\textbf{Выполнил:} \\
студент группы 3825М1ПМкн1 \\
Сучков В.Н.\\[0.3cm]
Подпись:\\[0.3cm]
\textbf{Научный руководитель:}\\
доцент каф. АГДМ, к.ф.-м.н.,\\  Сидоров С.В.\\[0.3cm]
Подпись:\\[0.3cm]
\end{flushleft}
\end{minipage}
\end{flushright}

\vfill

Нижний Новгород \\
2025

\thispagestyle{empty}
\end{center}

\newpage
\tableofcontents
\thispagestyle{empty}
\newpage
\setcounter{page}{1}
\section*{Введение}
\addcontentsline{toc}{section}{Введение}

Пусть есть матрицы $A, B \in \intmat{n}{n}$. Множество решений уравнения $AX = XB$ над $\ratmat{n}{n}$ 
обозначим как $M_{A,B} = \{X\in \ratmat{n}{n} |  AX = XB \}$. Тогда множество целочисленных решений есть $\Lambda_{A,B} = M_{A,B} \cap \intmat{n}{n}$.
Суть настоящей работы заключается в изучении структуры множества $\Lambda_{A,B}$, а в особенности, его связь с множестовм $M_{J,J}$, где 
$J$ - жорданова нормальная форма матриц $A$ и $B$.

\newpage
\section*{Свойства решений уравнения}
\addcontentsline{toc}{section}{Свойства решений уравнения}

Опишем несколько полезных свойств матриц из $M_{A,B}$, которые окажутся полезными в дальнейшем. Пусть $J$ - рациональная жорданова нормальная форма 
матриц $A$ и $B$ обладающих целочисленным спектром. Обозначим $S_A$ и $S_B$ соотвествующие трансформирующие матрицы над $\mathbb{Q}$, такие что $AS_A = S_AJ$ и $BS_B = S_BJ$. 
Известно, что $M_{A,B} = S_AM_{J,J}S_B^{-1}$ (Лемма 1 из \cite{Sidorov1}). Это равенство устанавливает связь между множествами $M_{A,B}$ и $M_{J,J}$, 
элементы последнего из которых суть решения уравнения $JX = XJ$.

\begin{definition} \label{def:1}
    Матрицу $T$ размера $m \times n$ будем называть обобщенно-верхне-треугольной, если она имеет один из следующих видов:
    \begin{enumerate}[label={\arabic*)}]
        \item Если $m = n$, $T$ -- квадратная верхне-треугольная матрица;
        \item Если $m < n$, $T$ = $\left(\begin{matrix}
            0 & T'\\
        \end{matrix}\right)$, где $T'$ -- квадратная верхне-треугольная матрица порядка $m$;
        \item Если $m > n$, $T$ = $\left(\begin{matrix}
            T'\\
            0\\
        \end{matrix}\right)$, где $T'$ -- квадратная верхне-треугольная матрица порядка $n$;
    \end{enumerate}

    Если в каждом из трех случаев у верхне-треугольной матрицы любая наддиагональ состоит из одинаковых элементов, то такую 
    обобщенно-верхне-треугольную матрицу назовем \textbf{правильной верхне-треугольной}.
\end{definition}

Матрицы $ X\in M_{J,J}$ имеют блочно-диагональный вид (\cite{Gantmacher}, с. 199)
$X = diag(X^{(1)}, \dots, X^{(s)})$, где $s$ -- число различных собственных чисел $\alpha_k$, $k = \overline{1, s}$ матриц $A$ и $B$. Каждый из 
диагональных блоков сам по себе имеет блочную структуру. Блоки матрицы $X^{(k)}$ обозначим как $X^{(k)}_{ij}$, $i,j = \overline{1,p_k}$, где $p_k$ 
- число жордановых клеток, соотвествующих собственному числу $\alpha_k$. Каждый блок $X^{(k)}_{ij}$ имеем правильный верхне-треугольный вид.


\begin{lemm}
    Пусть $X$ - блочная матрица с блоками $X_{ij}$ размера $t_i \times t_j$, $i, j = \overline{1, m}$  имеющими 
    правильную верхне-треугольную структуру с элементами, стоящими на наддиагонали $s$ соответственно равными $x_{ij}^{(s)}$ ($s = 1$ -- главная диагональ) и пусть размеры блоков $t_k$ упорядочены следующим образом: $t_1 = t_2 = \dots t_{m_1} > 
    t_{m_1 + 1} = t_{m_1 + 2} = \dots t_{m_2} > \dots > t_{m_{\beta - 1} + 1} \dots t_{m_{\beta}}$. Тогда определитель 
    $\det X$ имеет следующий вид:
    $$ \det X = D_{1}^{t_{m_1}} \cdot D_{2}^{t_{m_2}} \cdot \dots \cdot D_{\beta}^{t_{m_{\beta}}}$$,
    где $D_{k}$ -- миноры матрицы $X$ порядка $m_{k}$ вида:
    $$ D_{k} = \left|\begin{matrix}
        x_{m_{k-1} + 1,m_{k-1} + 1}^{(1)} & x_{m_{k-1} + 1,m_{k-1} + 2}^{(1)} & \dots & x_{m_{k-1} + 1, m_{k-1} + m_{k}}^{(1)}\\
        x_{m_{k-1} + 2,m_{k-1} + 1}^{(1)} & x_{m_{k-1} + 2,m_{k-1} + 2}^{(1)} & \dots & x_{m_{k-1} + 2, m_{k-1} + m_{k}}^{(1)}\\
        \vdots & \vdots & \ddots & \vdots\\
        x_{m_{k-1} + m_{k},m_{k-1} + 1}^{(1)} & x_{m_{k},m_{k-1} + 2}^{(1)} & \dots & x_{m_{k-1} + m_{k}, m_{k-1} + m_{k}}^{(1)}\\
    \end{matrix}\right|$$

    и $m_0 = 1$.
\end{lemm}
\textit{Доказательство}. Разложим определитель матрицы $X$ используя теорему Лапласа следующим образом: 
выберем столбцы с номерами $1, 1 + t_1, \dots, 1 + \displaystyle \sum_{k = 1}^{m}{t_k}$, и строки с аналогичными номерами. 
Полученный на пересечении выбранных строк и столбцов минор будет единственным отличным от нуля в сумме определителя. В самом деле,
Зафиксируем выбор столбцов и рассмотрим, как будут выглядеть миноры построенные на других строках. В сущности, в каждом из блоков мы 
выбрали первый столбец и можем выбирать строки. Всего возможно несколько видов блоков, в соответсвии с определением \ref{def:1}. Рассмотрим
произвольный блок $X_{ij}$. Если $t_i < t_j$, то блок имеет вид $\left(\begin{matrix} 0 & X_{ij}'\\ \end{matrix}\right)$
В таком случае, очевидно, что первый столбец целиком будет состоять из нулей, а значит при любом выборе строк в минор на соотвествующие позиции
будут поставлены нули. Если $t_i = t_j$, то $X_{ij}$ -- квадратная матрица верхне-треугольного вида, а значит, все элементы первого столбца, кроме, возможно, первого, 
буду равны нулю. И, наконец, если $t_i > t_j$ то блок $X_{ij} = \left(\begin{matrix} X_{ij}'\\ 0\\ \end{matrix}\right)$ и $X_{ij}'$ - квадратная верхне-треугольная матрица.
В матрице $X_{ij}'$, как мы уже выяснили, все элементы кроме первого в первом столбце обязательно равны нулю, а в оставшейся подматрице и вовсе
все элементы равны нулю. Таким образом, любой минор, стоящий на пересечении обозначеных выше столбцов и произвольных строк содержащий
хотя бы одну строку отличную от строк с номерами $1, 1 + t_1, \dots, 1 + \displaystyle \sum_{k = 1}^{m}{t_k}$ будет содержать тем самым
нулевую строку, а значит будет равен нулю.


Обозначим $M$ полученный таким образом минор и $M_1$ дополнительный к нему минор.
Тогда $\det X = M\cdot M_1$.


Покажем, что $M = D_1 \cdot D_2 \cdot \dots \cdot D_{\beta}$.
Рассмотрим среди строк с номерами $1, 1 + t_1, \dots, 1 + \displaystyle \sum_{i = 1}^{m}{t_i}$ некоторую $s$-ую строку матрицы $X$ соотвествующую блочной строке с номером $k$ ($X_{kl}, l = \overline{1, m}$).
Пусть $k_1$ наибольшее такое, что $t_{k_1} > t_k$.
Тогда, в соответсвии с определением \ref{def:1} элементы, стоящие в строке $s$ на позициях с номерами $1, 1 + t_1, \dots, 1 + \displaystyle \sum_{i = 1}^{k_1 - 1}$ будут равны
нулю, так как они взяты из блоков вида $X_{kl} = \left(\begin{matrix} 0 & X_{kl}'\\ \end{matrix}\right), l=\overline{1, k_1}$ (в силу того, что $t_k < t_l$), в которых элемент, стоящий
на пересечении первой строки и первого столбца равен нулю. Начиная с позиции $p = 1 + \displaystyle \sum_{i = 1}^{k_1 + 1}{t_i}$ соотвествующие блоки, содержащие элемент стоящий на пересечении $s$-ой строки и
$p$-го столбца имеют один из двух оставшихся видов, а значит тот элемент, в общем случае, отличен от нуля. Если такого номера $k_1$ нет, то такая строка не обязательно начинается с нулей. Таким образом, минор $M$ имеет ступенчатый вид:
$$M = \left|\begin{array}{c c c | c c c | c c c | c}
    x_{11}^{(0)} & \dots & x_{1m_1}^{(0)} & \dots & \dots & \dots & & & &\\
    x_{21}^{(0)} & \dots & x_{2m_1}^{(0)} & \dots & \dots & \dots & & & &\\
    \vdots & \vdots & \ddots & \vdots & \vdots & \vdots & & & &\\
    x_{m_1,1}^{(0)} & \dots & x_{m_1,m_1}^{(0)} & \dots & \dots & \dots & & &\\
    \hline
    0 & \dots & 0 & x_{m_1 + 1, m_1 + 1}^{(0)} & \dots & x_{m_1 + 1, m_2}^{(0)} & \dots & \dots & \dots & \vdots\\
    0 & \dots & 0 & x_{m_1 + 2, m_1 + 1}^{(0)} & \dots & x_{m_1 + 2, m_2}^{(0)} & \dots & \dots & \dots & \vdots\\
    \vdots & \vdots & \vdots & \vdots & \ddots & \vdots & \vdots & \vdots & \vdots & \vdots \\
    0 & \dots & 0 & x_{m_2, m_1 + 1}^{(0)} & \dots & x_{m_2, m_2}^{(0)} & \dots & \dots & \dots & \vdots\\
    \hline
    0 & \dots & 0 & 0 & \dots & 0 &  x_{m_2 + 1, m_2 + 1}^{(0)} & \dots & x_{m_2 + 1, m_3}^{(0)} & \vdots\\
    0 & \dots & 0 & 0 & \dots & 0 &  x_{m_2 + 2, m_2 + 1}^{(0)} & \dots & x_{m_2 + 2, m_3}^{(0)} & \vdots\\
    \vdots & \vdots & \vdots & \vdots & \vdots & \vdots & \vdots & \ddots & \vdots & \vdots \\
    0 & \dots & 0 & 0 & \dots & 0 &  x_{m_3, m_2 + 1}^{(0)} & \dots & x_{m_3, m_3}^{(0)} & \vdots\\
    \hline
    0 & \dots & 0 & 0 & \dots & 0 &  0 & \dots & 0 & \ast\\

\end{array}\right|$$

Заметим, что блоки стоящие на диагонали и есть $D_1, \dots, D_{\beta}$. В самом деле, исходя из того, как мы упорядочили величины $t_k$, количество нулей в начале строки
$k$-го диагонального блока будет равно $m_{k-1}$. Применяя теорему Лапласа к этому минору и получим выражение $M = D_1 \cdot D_2 \cdot \dots \cdot D_{\beta}$.

Теперь рассмотрим вид дополнительного минора $M_1$. Этот минор получен из матрицы $X$ вычеркиванием строк и столбцов с номерами
$1, 1 + t_1, \dots, 1 + \displaystyle \sum_{i = 1}^{m}{t_i}$. То есть, в каждом блоке мы вычеркиваем первую строку и первый столбец. Полученная таким образом
матрица будет иметь такую же структуру, как и исходная матрица, однако каждый её блок станет меньше, или вовсе будет окончательно удален.
Таким образом, к минору $M_1$ можно преминить тот же подход, что мы применяли к $\det X$ изначально, получив $M_1 = M^{(1)} \cdot M_2$. Минор $M^{(1)}$ будет
также как и минор $M$ раскладываться в произведение миноров $D_1, \dots, D_{\beta}$, однако, начиная с $t_{m_{\beta}} - 1$, минор с меньшим порядком $D_{\beta}$
пропадет из $M^{(t_{m_{\beta}})}$, начиная с $t_{m_{\beta - 1}} - 1$ разложение покинет минор $D_{\beta - 1}$ и так далее.
В результате, подставляя $M_l = M^{(l)} \cdot M_{l + 1}$ и раскрывая значения соотвествующих миноров,
мы получим $\det X = M \cdot M_1 = M \cdot M^{(1)} \cdot M_2 = \dots = D_1^{t_{m_1}} \cdot D_2^{t_{m_2}} \cdot \dots \cdot D_{\beta}^{t_{m_{\beta}}}$.
Лемма доказана.



\newpage
\begin{thebibliography}{99}

\bibitem{Sidorov1}
    С. В. Сидоров, “О подобии матриц с целочисленным спектром над кольцом целых чисел”, 
    Изв. вузов. Матем., 2011, № 3, 86–94; Russian Math. (Iz. VUZ), 55:3 (2011), 77–84. https://doi.org/10.3103/S1066369X11030091
\bibitem{Gantmacher}
    Гантмахер Ф.Р., \emph{Теория матриц}, Наука, М., 1967

\end{thebibliography}
\end{document}